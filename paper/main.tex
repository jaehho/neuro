\documentclass[11pt]{article}
\usepackage[utf8]{inputenc}
\usepackage[T1]{fontenc}
\usepackage{amsmath, amssymb, amsfonts, amsthm}
\usepackage{geometry}
\usepackage{setspace}
\usepackage{graphicx}
\usepackage[style=numeric-comp, backend=biber, sorting=none]{biblatex}

% Layout settings for readability
\geometry{a4paper, margin=1in}
\linespread{1.15}

% Bibliography resource
\addbibresource{references.bib}

% Theorem/Definition styles
\theoremstyle{definition}
\newtheorem{definition}{Definition}

\begin{document}

\section{Microscopic Foundation: Neural Dynamics}

Before deriving the macroscopic plasticity rules, we must first establish the microscopic dynamics of the individual neurons comprising the network. We employ the Leaky Integrate-and-Fire (LIF) model, a standard reduction of the Hodgkin-Huxley formalism that captures the essential sub-threshold integration and thresholding behavior of cortical neurons \parencite{gerstner_neuronal_2014}.

\subsection{Membrane Potential Dynamics}
The state of a postsynaptic neuron $i$ is described by its membrane potential $V_i(t)$. In the absence of input, the membrane potential relaxes to a resting potential $E_L$. The evolution of $V_i(t)$ is governed by the conservation of current across the cell membrane, modeled as an RC circuit consisting of a leakage resistor $R_m$ and a membrane capacitor $C_m$ in parallel:
% cite where this information came from

\begin{equation}
    \tau_m \frac{dV_i(t)}{dt} = -(V_i(t) - E_L) + R_m I_{syn, i}(t) + R_m I_{ext, i}(t),
\end{equation}
where $\tau_m = R_m C_m$ represents the membrane time constant, typically in the range of 10--20 ms \parencite{dayan_theoretical_2001}. The term $I_{syn, i}(t)$ represents the total synaptic current received from presynaptic neurons, and $I_{ext, i}(t)$ accounts for any external background currents or noise.
% I like the inclusion of the typical range for the membrane time constant, it adds context.

\subsection{Synaptic Interaction}
The synaptic current $I_{syn, i}(t)$ is determined by the activity of the presynaptic population. Let the spike train of a presynaptic neuron $j$ be denoted by a sum of Dirac delta functions, $\rho_j(t) = \sum_k \delta(t - t_j^k)$. 
% point reader to later section on spike generation

The arrival of a spike from neuron $j$ induces a transient change in the conductance or current of neuron $i$. In the current-based approximation, appropriate for this level of reduction, the total synaptic current is the linear sum of filtered presynaptic spikes weighted by the synaptic efficacy $w_{ij}$:
% appropriate for what level of reduction? don't mention reduction in the model definition section. Where is this current-based approximation from?
\begin{equation}
    I_{syn, i}(t) = \sum_{j} w_{ij} \int_{-\infty}^{t} \alpha(t - s) \rho_j(s) \, ds,
\end{equation}
where $\alpha(t)$ is the postsynaptic current (PSC) kernel, typically modeled as an exponential decay $\alpha(t) = \frac{1}{\tau_s} e^{-t/\tau_s} \Theta(t)$, with synaptic time constant $\tau_s$.
% typically modeled according to who?

\subsection{Spike Generation Mechanism}
The continuous voltage dynamics defined above give rise to discrete events. A spike is generated at time $t_i^k$ when the membrane potential crosses a fixed threshold $\vartheta$ from below:
\begin{equation}
    t_i^k : V_i(t_i^k) = \vartheta \quad \text{and} \quad \frac{dV_i}{dt}\bigg|_{t=t_i^k} > 0.
\end{equation}
Immediately following a spike, the potential is reset to a value $V_{reset} < \vartheta$ and held constant for a refractory period $\tau_{ref}$, simulating the temporary inactivation of $Na^+$ channels. 
% If the potential is reset after a spike, it would never cross the threshold from above, so there is either no need to mention that condition or change it.

This mechanism defines the postsynaptic spike train $\rho_i(t) = \sum_k \delta(t - t_i^k)$, which serves as the input to the plasticity equations in the following section.

\section{Mathematical Formulation of the Three-Factor Plasticity Model}

Having defined the generation of spike times $\mathcal{T}_j$ and $\mathcal{T}_i$ via the LIF dynamics, we now analyze the evolution of the synaptic weights $w_{ij}$. We focus on a plasticity rule belonging to the class of \textit{three-factor learning rules}, as reviewed by \textcite{fremaux_neuromodulated_2016}. In this framework, synaptic updates are gated by a global neuromodulatory signal (factor three) rather than relying solely on the pairwise correlation of presynaptic and postsynaptic activity (factors one and two).

\subsection{Neural Activity and Notation}

Following the framework established in the previous section, the neural response functions are formally treated as sums of Dirac distributions:
\begin{equation}
    \rho_j(t) = \sum_{k=1}^{N_j} \delta(t - t_j^k) \quad \text{and} \quad \rho_i(t) = \sum_{k=1}^{N_i} \delta(t - t_i^k).
\end{equation}
Here, $N_j$ and $N_i$ denote the total number of spikes fired by each neuron over the interval $t \in [0, T]$, determined stochastically by the interaction of the membrane potential equation (Eq. 1) and the threshold condition (Eq. 3).
% Use latex equation referencing. Elaborate on what you mean by "determined stochastically..."

To prevent unbounded growth, we constrain the weight to a closed interval $w_{ij} \in [0, w_{\text{max}}]$, where $w_{\text{max}} \in \mathbb{R}^+$ is a constant parameter representing the maximum possible synaptic efficacy \parencite{fremaux_neuromodulated_2016}.
% What is a typical value for w_max, where does this constraint come from?

\subsection{Local Dynamics: The Eligibility Trace}
A central feature of this model is that the coincidence of spikes does not immediately change the synaptic weight. Instead, it creates a temporary memory trace, $E_{ij}(t)$, known as the \textit{eligibility trace} \parencite{gerstner_neuronal_2014, fremaux_neuromodulated_2016}. This trace allows the synapse to bridge the temporal delay—often termed the "distal reward problem"—between millisecond-scale neural activity and second-scale reward signals. The eligibility trace evolves according to:
\begin{equation}
    \tau_e \frac{dE_{ij}(t)}{dt} = -E_{ij}(t) + S_{ij}(t),
\end{equation}
where $\tau_e \in \mathbb{R}^+$ is the decay time constant of the trace.
% what are the typical values for tau_e, where does this equation come from?

The driving term $S_{ij}(t)$ represents the instantaneous induction of Spike-Timing-Dependent Plasticity (STDP). To define $S_{ij}(t)$, we use variables $x_j(t)$ and $y_i(t)$ that track the recent history of presynaptic and postsynaptic activity, respectively:
\begin{align}
    \tau_+ \frac{dx_j(t)}{dt} &= -x_j(t) + \rho_j(t), \\
    \tau_- \frac{dy_i(t)}{dt} &= -y_i(t) + \rho_i(t),
\end{align}
where $\tau_+, \tau_- \in \mathbb{R}^+$ are the time constants for the potentiation and depression windows. These constants are derived from experimental data and define the temporal window of sensitivity for plasticity \parencite{dayan_theoretical_2001}.
% what are typical values for tau_+ and tau_-?

The STDP induction term $S_{ij}(t)$ combines Long-Term Potentiation (LTP) and Long-Term Depression (LTD):
\begin{equation}
    S_{ij}(t) = \underbrace{A_+(w_{ij}) x_j(t) \rho_i(t)}_{\text{LTP contribution}} - \underbrace{A_-(w_{ij}) y_i(t) \rho_j(t)}_{\text{LTD contribution}}.
\end{equation}
The LTP term is active only when a postsynaptic spike occurs ($\rho_i(t) \neq 0$), and its magnitude depends on the accumulated presynaptic activity $x_j(t)$. Similarly, the LTD term is active when a presynaptic spike occurs, depending on the postsynaptic activity.

\subsection{Weight Dependence and Stability}
To ensure the weight $w_{ij}$ stays within the bounds $[0, w_{\text{max}}]$, the scaling functions $A_+(w_{ij})$ and $A_-(w_{ij})$ include a "soft bound" dependence on the current weight:
\begin{equation}
    A_+(w_{ij}) = \eta_+ (w_{\text{max}} - w_{ij}) \quad \text{and} \quad A_-(w_{ij}) = \eta_- w_{ij},
\end{equation}
where $\eta_+, \eta_- \in \mathbb{R}^+$ are the learning rates. This formulation ensures that the rate of weight change drops to zero as the weight approaches either limit.
% where do these equations come from, what are typical values for eta_+ and eta_-?

\subsection{Global Dynamics: Neuromodulated Update}
The actual change in synaptic weight is governed by a global neuromodulatory signal $M(t)$. The differential equation for the weight is the product of this global signal and the local eligibility trace:
\begin{equation}
    \frac{dw_{ij}(t)}{dt} = M(t) E_{ij}(t).
\end{equation}
The signal $M(t)$ is modeled as a Reward Prediction Error (RPE), calculated as the difference between the instantaneous reward $R(t)$ and a baseline expectation $\bar{R}(t)$:
\begin{equation}
    M(t) = R(t) - \bar{R}(t).
\end{equation}
Here, $\bar{R}(t)$ serves as a reference point. It ensures that the neuromodulatory signal can be positive (indicating better-than-expected outcomes) or negative (indicating worse-than-expected outcomes), allowing for bidirectional regulation of synaptic weights \parencite{fremaux_neuromodulated_2016}.

\printbibliography

\end{document}