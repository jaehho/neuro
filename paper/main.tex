\documentclass[11pt]{article}
\usepackage[utf8]{inputenc}
\usepackage[T1]{fontenc}
\usepackage{amsmath, amssymb, amsfonts, amsthm}
\usepackage{geometry}
\usepackage{setspace}
\usepackage{graphicx}
\usepackage{booktabs}   % Professional table formatting
\usepackage{csquotes}

% Biblatex setup
\usepackage[style=numeric-comp, backend=biber, sorting=none]{biblatex}
\addbibresource{references.bib}

% Hyperref should generally be loaded last
\usepackage[colorlinks=true, linkcolor=blue, citecolor=blue, urlcolor=blue]{hyperref}

% Layout settings for readability
\geometry{a4paper, margin=1in}
\linespread{1.15}

% Theorem/Definition styles
\theoremstyle{definition}
\newtheorem{definition}{Definition}

\begin{document}

\section{Microscopic Foundation: Neural Dynamics}

We consider a minimal circuit consisting of a single presynaptic neuron connected to a single postsynaptic neuron through a unidirectional synapse of weight $w$. This section specifies the dynamics governing the postsynaptic membrane potential, the synaptic coupling, and the generation of spikes.

\subsection{Membrane Potential Dynamics}
The postsynaptic neuron is modeled as a Leaky Integrate-and-Fire (LIF) unit, a standard reduction of the Hodgkin--Huxley formalism that retains sub-threshold integration and thresholding \parencite{gerstner_neuronal_2014}. Its membrane potential $V(t)$ evolves according to the conservation of current across the cell membrane, modeled as a parallel RC circuit with leakage resistance $R_m$ and membrane capacitance $C_m$ \parencite{dayan_theoretical_2001}:
\begin{equation} \label{eq:lif_voltage}
    \tau_m \frac{dV(t)}{dt} = -(V(t) - E_L) + R_m I_{\mathrm{syn}}(t) + R_m I_{\mathrm{ext}}(t),
\end{equation}
where $\tau_m = R_m C_m$ is the membrane time constant, $E_L$ is the resting potential, $I_{\mathrm{syn}}(t)$ is the synaptic current from the presynaptic neuron, and $I_{\mathrm{ext}}(t)$ is an external current, typically modeled as an injected current or Gaussian white noise.

\subsection{Spike Generation} \label{sec:spike_gen}
A spike is defined as the instant $t_{\mathrm{post}}^k$ at which the membrane potential crosses a fixed threshold $\theta$ from below:
\begin{equation} \label{eq:threshold}
    t_{\mathrm{post}}^k : \quad V(t_{\mathrm{post}}^k) = \theta \quad \text{and} \quad \left.\frac{dV}{dt}\right|_{t=t_{\mathrm{post}}^k} > 0.
\end{equation}
The derivative condition distinguishes the upward threshold crossing from subsequent repolarization. Upon firing, $V$ is reset to $V_{\mathrm{reset}} < \theta$ and the dynamics in \eqref{eq:lif_voltage} are suspended for a refractory period $\tau_{\mathrm{ref}}$, after which integration resumes from $V_{\mathrm{reset}}$ \parencite{dayan_theoretical_2001}.

The spike trains of both neurons are written as sums of Dirac distributions:
\begin{equation} \label{eq:spike_trains}
    \rho_{\mathrm{pre}}(t) = \sum_{k} \delta(t - t_{\mathrm{pre}}^k), \qquad
    \rho_{\mathrm{post}}(t) = \sum_{k} \delta(t - t_{\mathrm{post}}^k).
\end{equation}
The presynaptic spike times $\{t_{\mathrm{pre}}^k\}$ are taken as given (e.g., drawn from a Poisson process), while the postsynaptic times $\{t_{\mathrm{post}}^k\}$ are determined by \eqref{eq:lif_voltage} and \eqref{eq:threshold}.

\subsection{Synaptic Interaction}
The synaptic current is determined by the presynaptic spike train filtered through a postsynaptic current (PSC) kernel and scaled by the synaptic weight $w$. Under the \textit{current-based} approximation, which treats synaptic currents as independent of the postsynaptic membrane potential, the synaptic current is \parencite{gerstner_neuronal_2014}:
\begin{equation} \label{eq:synaptic_current}
    I_{\mathrm{syn}}(t) = w \int_{-\infty}^{t} \alpha(t - s)\, \rho_{\mathrm{pre}}(s) \, ds,
\end{equation}
where $\alpha(t) = \tau_s^{-1} e^{-t/\tau_s}\, \Theta(t)$ is an exponential PSC kernel with synaptic time constant $\tau_s$ and $\Theta(t)$ the Heaviside step function \parencite{dayan_theoretical_2001}.

\section{Three-Factor Plasticity Model}

We now turn to the evolution of the synaptic weight $w$. The plasticity rule belongs to the class of \textit{three-factor learning rules} reviewed by \textcite{fremaux_neuromodulated_2016}. In standard Spike-Timing-Dependent Plasticity (STDP), weight changes depend on the correlation of pre- and postsynaptic spike times; three-factor rules gate this local signal with a global neuromodulatory factor representing reward or error feedback.

\subsection{Weight-Dependent Scaling and Stability} \label{sec:weight_constraints}
Before specifying the plasticity dynamics, we define the weight-dependent scaling functions that appear in the learning rule. The weight $w$ represents the efficacy of the synapse from the presynaptic to the postsynaptic neuron and is constrained to the interval $[0, w_{\mathrm{max}}]$, where $w_{\mathrm{max}}$ is a physiological saturation limit.

The amplitudes of potentiation and depression are modulated by soft-bound functions \parencite{gerstner_neuronal_2014}:
\begin{equation} \label{eq:soft_bounds}
    A_+(w) = \eta_+ (w_{\mathrm{max}} - w), \qquad A_-(w) = \eta_- w,
\end{equation}
where $\eta_+$ and $\eta_-$ are learning rates. These linear dependencies cause the rate of weight change to diminish as $w$ approaches either boundary. This prevents unbounded growth of the weight and biases the dynamics toward the interior of $[0, w_{\mathrm{max}}]$. However, because the full weight update (defined below in \eqref{eq:weight_update}) involves a signed modulation signal, the soft bounds alone do not strictly guarantee $w \in [0, w_{\mathrm{max}}]$; in practice, a hard clipping step $w \leftarrow \max(0, \min(w, w_{\mathrm{max}}))$ may be applied after each update.

\subsection{Local Dynamics: The Eligibility Trace}
A central feature of this model is that coincident pre- and postsynaptic spikes create a temporary memory called the \textit{eligibility trace} $E(t)$ \parencite{fremaux_neuromodulated_2016}. This trace allows the synapse to bridge the temporal gap between millisecond-scale neural activity and delayed reward signals. It evolves as:
\begin{equation} \label{eq:eligibility}
    \tau_e \frac{dE(t)}{dt} = -E(t) + S(t),
\end{equation}
where $\tau_e$ is a decay time constant, typically on the order of hundreds of milliseconds to seconds for reinforcement learning tasks \parencite{gerstner_neuronal_2014}.

The driving term $S(t)$ captures the instantaneous STDP induction. To define it, we introduce filtered spike-history variables for each neuron:
\begin{align}
    \tau_+ \frac{dx_{\mathrm{pre}}(t)}{dt} &= -x_{\mathrm{pre}}(t) + \rho_{\mathrm{pre}}(t), \label{eq:pre_trace} \\
    \tau_- \frac{dy_{\mathrm{post}}(t)}{dt} &= -y_{\mathrm{post}}(t) + \rho_{\mathrm{post}}(t), \label{eq:post_trace}
\end{align}
where $\tau_+$ and $\tau_-$ set the widths of the potentiation and depression windows, respectively. Experimental measurements place these values in the range 20--40\,ms \parencite{bi_synaptic_1998}. The STDP induction term then combines Long-Term Potentiation (LTP) and Long-Term Depression (LTD):
\begin{equation} \label{eq:stdp_induction}
    S(t) = \underbrace{A_+(w)\, x_{\mathrm{pre}}(t)\, \rho_{\mathrm{post}}(t)}_{\text{LTP}} \;-\; \underbrace{A_-(w)\, y_{\mathrm{post}}(t)\, \rho_{\mathrm{pre}}(t)}_{\text{LTD}},
\end{equation}
with $A_+(w)$ and $A_-(w)$ as defined in \eqref{eq:soft_bounds}.

\subsection{Global Dynamics: Neuromodulated Update}
The weight evolves under the product of the eligibility trace and a global neuromodulatory signal $M(t)$:
\begin{equation} \label{eq:weight_update}
    \frac{dw(t)}{dt} = M(t)\, E(t).
\end{equation}

To construct $M(t)$, we first define smooth estimates of the instantaneous firing rates by low-pass filtering the spike trains with a rate time constant $\tau_r$:
\begin{align}
    \tau_r \frac{dr_{\mathrm{pre}}(t)}{dt} &= -r_{\mathrm{pre}}(t) + \rho_{\mathrm{pre}}(t), \label{eq:rate_pre} \\
    \tau_r \frac{dr_{\mathrm{post}}(t)}{dt} &= -r_{\mathrm{post}}(t) + \rho_{\mathrm{post}}(t). \label{eq:rate_post}
\end{align}
Unlike a rectangular sliding-window estimator, these exponentially weighted averages are smooth functions of time and consistent in form with the other filtered quantities in the model.

As a simplified stand-in for a more general objective, we define the instantaneous reward as a penalty on the deviation of the postsynaptic rate from half the presynaptic rate:
\begin{equation} \label{eq:reward}
    R(t) = -\left(r_{\mathrm{post}}(t) - \tfrac{1}{2}\, r_{\mathrm{pre}}(t)\right)^2.
\end{equation}
The modulation signal $M(t)$ is then a Reward Prediction Error (RPE), computed as the difference between $R(t)$ and a slowly adapting baseline $\bar{R}(t)$ \parencite{fremaux_neuromodulated_2016}:
\begin{equation} \label{eq:rpe}
    M(t) = R(t) - \bar{R}(t),
\end{equation}
where $\bar{R}(t)$ tracks the running average of the reward:
\begin{equation} \label{eq:reward_baseline}
    \tau_{\bar{R}} \frac{d\bar{R}(t)}{dt} = -\bar{R}(t) + R(t).
\end{equation}
The baseline enables bidirectional regulation: performance better than expected yields $M > 0$ (reinforcing the current eligibility trace), while performance worse than expected yields $M < 0$ (weakening it).

\subsection{Summary of Symbols}

\begin{table}[h]
\centering
\caption{Summary of notation.}
\label{tab:symbols}
\begin{tabular}{@{}lll@{}}
\toprule
Symbol & Description & Equation \\
\midrule
$V(t)$              & Postsynaptic membrane potential              & \eqref{eq:lif_voltage} \\
$E_L$               & Resting (leak) potential                     & \eqref{eq:lif_voltage} \\
$\theta$            & Spike threshold                              & \eqref{eq:threshold} \\
$V_{\mathrm{reset}}$ & Reset potential after spike                  & \S\ref{sec:spike_gen} \\
$\rho_{\mathrm{pre}}(t),\; \rho_{\mathrm{post}}(t)$ & Spike trains (sums of Dirac deltas) & \eqref{eq:spike_trains} \\
$\alpha(t)$         & Postsynaptic current kernel                  & \eqref{eq:synaptic_current} \\
$I_{\mathrm{syn}}(t)$ & Total synaptic current                     & \eqref{eq:synaptic_current} \\
$I_{\mathrm{ext}}(t)$ & External / noise current                   & \eqref{eq:lif_voltage} \\
$w$                 & Synaptic weight                              & \eqref{eq:weight_update} \\
$A_+(w),\; A_-(w)$  & Weight-dependent LTP/LTD amplitudes          & \eqref{eq:soft_bounds} \\
$x_{\mathrm{pre}}(t)$ & Presynaptic spike-history trace            & \eqref{eq:pre_trace} \\
$y_{\mathrm{post}}(t)$ & Postsynaptic spike-history trace          & \eqref{eq:post_trace} \\
$S(t)$              & STDP induction term                          & \eqref{eq:stdp_induction} \\
$E(t)$              & Eligibility trace                            & \eqref{eq:eligibility} \\
$r_{\mathrm{pre}}(t),\; r_{\mathrm{post}}(t)$ & Exponentially filtered firing rates & \eqref{eq:rate_pre}--\eqref{eq:rate_post} \\
$R(t)$              & Instantaneous reward signal                  & \eqref{eq:reward} \\
$\bar{R}(t)$        & Reward baseline (running average)             & \eqref{eq:reward_baseline} \\
$M(t)$              & Neuromodulatory signal (RPE)                 & \eqref{eq:rpe} \\
\bottomrule
\end{tabular}
\end{table}

\subsection{Parameter Values}

\begin{table}[h]
\centering
\caption{Model parameters and representative values. Ranges are drawn from the cited experimental and modeling literature.}
\label{tab:parameters}
\begin{tabular}{@{}llll@{}}
\toprule
Parameter & Description & Typical value & Source \\
\midrule
$\tau_m$            & Membrane time constant         & 10--20\,ms       & \parencite{dayan_theoretical_2001} \\
$R_m$               & Membrane resistance            & 10--100\,M$\Omega$ & \parencite{dayan_theoretical_2001} \\
$C_m$               & Membrane capacitance           & $\tau_m / R_m$   & --- \\
$E_L$               & Resting potential              & $-70$\,mV        & \parencite{dayan_theoretical_2001} \\
$\theta$            & Spike threshold                & $-55$\,mV        & \parencite{dayan_theoretical_2001} \\
$V_{\mathrm{reset}}$ & Reset potential               & $-70$\,mV        & \parencite{dayan_theoretical_2001} \\
$\tau_{\mathrm{ref}}$ & Absolute refractory period   & 2--5\,ms         & \parencite{gerstner_neuronal_2014} \\
$\tau_s$            & Synaptic time constant (PSC)   & 2--10\,ms        & \parencite{destexhe_kinetic_1998} \\
$w_{\mathrm{max}}$  & Maximum synaptic weight        & model-dependent  & --- \\
$\eta_+$            & LTP learning rate              & model-dependent  & --- \\
$\eta_-$            & LTD learning rate              & model-dependent  & --- \\
$\tau_+$            & Potentiation window width      & 20--40\,ms       & \parencite{bi_synaptic_1998} \\
$\tau_-$            & Depression window width         & 20--40\,ms       & \parencite{bi_synaptic_1998} \\
$\tau_e$            & Eligibility trace decay        & 0.1--1.0\,s      & \parencite{gerstner_neuronal_2014} \\
$\tau_r$            & Rate-estimation time constant   & 50--200\,ms      & --- \\
$\tau_{\bar{R}}$    & Reward baseline time constant   & 1--10\,s         & \parencite{fremaux_neuromodulated_2016} \\
\bottomrule
\end{tabular}
\end{table}

\printbibliography

\end{document}